\documentclass[12pt]{report}
\usepackage[utf8]{inputenc}
\usepackage[russian]{babel}
%\usepackage[14pt]{extsizes}
\usepackage{listings}

% Для листинга кода:
\lstset{ %
language=go,                 % выбор языка для подсветки 
basicstyle=\small\sffamily, % размер и начертание шрифта для подсветки кода
numbers=left,               % где поставить нумерацию строк (слева\справа)
numberstyle=\tiny,           % размер шрифта для номеров строк
stepnumber=1,                   % размер шага между двумя номерами строк
numbersep=5pt,                % как далеко отстоят номера строк от подсвечиваемого кода
showspaces=false,            % показывать или нет пробелы специальными отступами
showstringspaces=false,      % показывать или нет пробелы в строках
showtabs=false,             % показывать или нет табуляцию в строках            
tabsize=2,                 % размер табуляции по умолчанию равен 2 пробелам
captionpos=t,              % позиция заголовка вверху [t] или внизу [b] 
breaklines=true,           % автоматически переносить строки (да\нет)
breakatwhitespace=false, % переносить строки только если есть пробел
escapeinside={\#*}{*)}   % если нужно добавить комментарии в коде
}

% Для измененных титулов глав:
\usepackage{titlesec, blindtext, color} % подключаем нужные пакеты
\definecolor{gray75}{gray}{0.75} % определяем цвет
\newcommand{\hsp}{\hspace{20pt}} % длина линии в 20pt
% titleformat определяет стиль
\titleformat{\chapter}[hang]{\Huge\bfseries}{\thechapter\hsp\textcolor{gray75}{|}\hsp}{0pt}{\Huge\bfseries}

%отступы по краям
\usepackage{geometry}
\geometry{verbose, a4paper,tmargin=2cm, bmargin=2cm, rmargin=1.5cm, lmargin = 3cm}
% межстрочный интервал
\usepackage{setspace}
\onehalfspacing
\usepackage{float}
% plot
\usepackage{pgfplots}
\usepackage{filecontents}
\usepackage{amsmath}
\usepackage{tikz,pgfplots}
\usetikzlibrary{datavisualization}
\usetikzlibrary{datavisualization.formats.functions}

\usepackage{graphicx}
\graphicspath{{src/}}
\DeclareGraphicsExtensions{.pdf,.png,.jpg}

\usepackage{geometry}
\geometry{verbose, a4paper,tmargin=2cm, bmargin=2cm, rmargin=1.5cm, lmargin = 3cm}
\usepackage{indentfirst}
\setlength{\parindent}{1.4cm}

\usepackage{titlesec}
\titlespacing{\chapter}{0pt}{12pt plus 4pt minus 2pt}{0pt}


\begin{document}
%\def\chaptername{} % убирает "Глава"
\begin{titlepage}
	\centering
	{\scshape\LARGE МГТУ им. Баумана \par}
	\vspace{3cm}
	{\scshape\Large Лабораторная работа №5\par}
	\vspace{0.5cm}	
	{\scshape\Large По курсу: "Операционные системы"\par}
	\vspace{1.5cm}
	{\huge\bfseries Процессы. Системные вызовы fork() и exec().\par}
	\vspace{2cm}
	\Large Работу выполнил: Мокеев Даниил, ИУ7-56\par
	\vspace{0.5cm}
	\Large Преподаватели:  Рязанова Н.Ю.\par

	\vfill
	\large \textit {Москва, 2019} \par
\end{titlepage}

\tableofcontents

\newpage
\chapter*{Введение}
\addcontentsline{toc}{chapter}{Введение}
Целью данной лабораторной работы является изучение системных вызовов fork() и exec(), программных каналов и сигналов, получение навыков их использования.

Задачи данной лабораторной работы:
\begin{itemize}
	\item изучить системные вызовы exec() и fork();
	\item рассмотреть программные каналы и сигналы;
	\item реализовать пять программ по заданию.
\end{itemize}

\chapter{Аналитическая часть}
В данной части будут рассмотрены теоретические основы задачи коммивояжера и муравьиного алгоритма. 

\section{Постановка задачи} 
Имеется сильно связный взвешенный ориентированный граф \cite{diskr} с положительными весами, заданный в виде матрицы смежностей. Количество вершин в нем лежит в диапазоне от 5 до 20.Требуется решить задачу коммивояжера для этого графа. 

\section*{Вывод}
\addcontentsline{toc}{section}{Введение}
В данном разделе были рассмотрены общие принципы муравьиного алгоритма и применение его к задаче коммивояжера. 


\chapter{Конструкторская часть}
В данном разделе будут рассмотрены основные требования к программе и схемы алгоритмов.

\section{Требования к программе}
\textbf{Требования к вводу:}
\begin{itemize}
	\item У ориентированного графа должно быть хотя бы 2 вершины.
\end{itemize}

\textbf{Требования к программе:}
\begin{itemize}
	\item Алгоритм полного перебора должен возвращать кратчайший путь в графе.
\end{itemize}
.  
\newline  
\textbf{Входные данные} - матрица смежности графа.  
\newline
\textbf{Выходные данные} - самый выгодный путь.


\section*{Вывод}
\addcontentsline{toc}{section}{Вывод}
В данном разделе были рассмотрены требования к программе и схемы алгоритмов.


\chapter{Технологическая часть}

\section{Листинг кода алгоритмов}
В данном разделе будут приведены листинги кода полного перебора всех решений (Листинг \ref{brute}) и реализации муравьиного алгоритма (Листинг \ref{ants})
\begin{lstlisting}[label=brute,caption = Перебор всех возможных вариантов, language = go]

func brute(file_name string) []int{
	weight := get_weights(file_name)
	path := make([]int, 0)
	res := make([]int, len(weight))

	for k:=0; k<len(weight);k++{
		ways := make([][]int, 0)
		_ = go_route(k, weight, path, &ways)
		sum := 1000
		curr := 0
		ind := 0
		for i:=0; i<len(ways);i++{
			curr = 0
			for j:=0;j<len(ways[i])-1;j++{
				curr+=weight[ways[i][j]][ways[i][j+1]]
			}
			if curr < sum{
				sum = curr
			}
		}
		res[k] = sum
	}
	return res
}

func contains(s []int, e int) bool {
	for _, a := range s {
		if a == e {
			return true
		}
	}
	return false
}


func go_route(pos int, weight [][]int, path[]int, routes *[][]int)[]int{
	path = append(path, pos)
	if len(path) < len(weight){
		for i:=0; i < len(weight); i++{
			if !(contains(path, i)){
				_ = go_route(i, weight,path, routes)
			}
		}
		}else{
			*routes = append(*routes, path)
	}
	return path
}
\end{lstlisting}

\begin{lstlisting}[label=ants,caption = Муравьиный алгоритм, language = go]

func start (env *env, days int) []int{
	shortest_dist := make([]int, len(env.weight))
	for n := 0; n < days; n++{
		for i:= 0; i< len(env.weight); i++{
			ant := env.new_ant(i)
			ant.ant_go()
			
			cur_dist := ant.get_distance()
			if (shortest_dist[i] == 0) || (cur_dist < shortest_dist[i]){
				shortest_dist[i] = cur_dist
			}
		}
	}
	return shortest_dist
}

func (ant *ant) ant_go(){
	for{
		prob := ant.count_probapility()
		choosen_path := choose_path(prob)
		if choosen_path == -1{
			break}
		ant.go_path(choosen_path)
		ant.renew_pheromon()
	}
}

func (ant *ant)count_probapility() []float64{
	res := make([]float64, 0);
	var d float64;
	var sum float64;
	for i, lenght := range ant.visited[ant.position]{
		if lenght != 0{
			d = math.Pow((float64(1)/float64(lenght)), ant.env.alpha) * math.Pow(ant.env.pheromon[ant.position][i], ant.env.betta)
			res = append(res, d)
			sum += d
		} else{
			res = append(res, 0)
		}
		}
		for _, lenght := range res{
			lenght = lenght / sum
	}
	return res
}

func choose_path(probab []float64) int{
	var sum float64
	for _, j := range probab{
		sum += j
	}
	r := rand.New(rand.NewSource(time.Now().UnixNano()))
	random_fl := r.Float64() * sum
	sum = 0
	for i , j := range probab{
		if random_fl > sum && random_fl<sum+j{
			return i
		} else{
		sum+=j
		}
	}
	return -1
}
\end{lstlisting}
\section*{Вывод}
\addcontentsline{toc}{section}{Вывод}
В данном разделе были рассмотрены основные сведения о модулях программы и листинг кода алгоритмов.


\chapter*{Заключение}
\addcontentsline{toc}{chapter}{Заключение}
В ходе лабораторной работы были изучены и реализованы алгоритмы решения задачи коммивояжера - полный перебор и муравьиный алгоритм.

Временной анализ показал, что неэффективно использовать полный перебор на графе размера больше 8.

\addcontentsline{toc}{chapter}{Список литературы}
\begin{thebibliography}{3}
	\bibitem{diskr} Белоусов А.И., Ткачев С.Б(2006). Дискретная математика, 4-е издание.
	\bibitem{commi2} Т.М. Товстик, Е.В. Жукова - Алгоритм приближенного решения задачи коммивояжера.
	\bibitem{commi} Задача коммивояжера[Электронный ресурс] - режим доступа http://mech.math.msu.su/~shvetz/54/inf/perl-problems/chCommisVoyageur.xhtml
	\bibitem{ant1} Муравьиные алгоритмы[Электронный ресурс] - режим доступа http://www.machinelearning.ru/wiki/index.php?title=%D0%9C%D1%83%D1%80%D0%B0%D0%B2%D1%8C%D0%B8%D0%BD%D1%8B%D0%B5_%D0%B0%D0%BB%D0%B3%D0%BE%D1%80%D0%B8%D1%82%D0%BC%D1%8B
	\bibitem{shtovba} Штовба С.Д. - Муравьиные алгоритмы.
	\bibitem{Beloysov} И. В. Белоусов(2006), Матрицы и определители, учебное пособие по линейной алгебре, с. 1 - 16
\end{thebibliography}

\end{document}
