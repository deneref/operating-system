\documentclass[12pt]{report}
\usepackage[utf8]{inputenc}
\usepackage[russian]{babel}
%\usepackage[14pt]{extsizes}
\usepackage{listings}
\usepackage{xcolor}

% Для измененных титулов глав:
\usepackage{titlesec, blindtext, color} % подключаем нужные пакеты
\definecolor{gray75}{gray}{0.75} % определяем цвет
\newcommand{\hsp}{\hspace{20pt}} % длина линии в 20pt
% titleformat определяет стиль
\titleformat{\chapter}[hang]{\Huge\bfseries}{\thechapter\hsp\textcolor{gray75}{|}\hsp}{0pt}{\Huge\bfseries}

%отступы по краям
\usepackage{geometry}
\geometry{verbose, a4paper,tmargin=2cm, bmargin=2cm, rmargin=1.5cm, lmargin = 3cm}
% межстрочный интервал
\usepackage{setspace}
\onehalfspacing
\usepackage{float}
% plot
\usepackage{pgfplots}
\usepackage{filecontents}
\usepackage{amsmath}
\usepackage{tikz,pgfplots}
\usetikzlibrary{datavisualization}
\usetikzlibrary{datavisualization.formats.functions}

\usepackage{graphicx}
\graphicspath{{src/}}
\DeclareGraphicsExtensions{.pdf,.png,.jpg}

\usepackage{geometry}
\geometry{verbose, a4paper,tmargin=2cm, bmargin=2cm, rmargin=1.5cm, lmargin = 3cm}
\usepackage{indentfirst}
\setlength{\parindent}{1.4cm}


\begin{document}
%\def\chaptername{} % убирает "Глава"
\begin{titlepage}
	\centering
	{\scshape\LARGE МГТУ им. Баумана \par}
	\vspace{3cm}
	{\scshape\Large Лабораторная работа №1 ч.2.\par}
	\vspace{0.5cm}	
	{\scshape\Large По курсу: "Операционные системы"\par}
	\vspace{1.5cm}
	\centering
	 {\huge\bfseries Функции обработчика системного таймера. Пересчет динамических приоритетов\par}
	\vspace{2cm}
	\Large Работу выполнил: Мокеев Даниил, ИУ7-56\par
	\vspace{0.5cm}
	\Large Преподаватель:  Рязанова Н.Ю.\par

	\vfill
	\large \textit {Москва, 2020} \par
\end{titlepage}


\newpage

\chapter{Функции обработчика прерываний от системного таймера}
\section{Unix}
\subsection{По тику}
\begin{itemize}
	\item Инкремент	счетчика использования процессора текущим процессом;
	\item инкремент часов и других таймеров системы;
	\item декремент счетчика времени, оставшегося до отправления на выполнение отложенных вызовов и отправка отложенных вызовов на выполнение, при достижении нулевого значения счетчика;
	\item декремент кванта.
\end{itemize}
\subsection{По главному тику}
\begin{itemize}
	\item	Добавление в очередь отложенных вызовов функций планировщик;
	\item	пробуждение системных процессов swapper и pagedaemon;
	
	\item	декремент счетчиков времени, оставшегося до отправления сигналов тревоги:
	\begin{itemize}
		\item	SIGALRM — сигнал будильника реального времени, который отравляется по истичении заданного промежутка реального времени;
		\item	SIGPROF — сигнал будильника профиля процесса, который измеряет время работы процесса;
		\item	SIGVTALRM — сигнал будильника виртуального времени,
		который измеряет время работы процесса в режиме задачи.
	\end{itemize}
\end{itemize}

\subsection{По кванту}

\begin{itemize}
	\item При превышении текущим процессом выделенного кванта, отправка сигнала SIGXCPU этому процессу.
\end{itemize}

\section{Windows}
\subsection{По тику}
\begin{itemize}
	\item Инкремент счетчика системного времени;
	\item декремент счетчиков отложенных задач;
	\item декремент остатка кванта текущего потока;
	\item активация обработчика ловушки профилирования ядра.
\end{itemize}
\subsection{По главному тику}
\begin{itemize}
	\item Инициализация диспетчера настройки баланса путем освобождения объекта «событие», на котором он ожидает.

\end{itemize}
\subsection{По кванту}
\begin{itemize}
	\item Инициализация диспетчеризации потоков путем добавления соответствующего объекта DPC в очередь.
\end{itemize}

\chapter{Пересчет динамических приоритетов}
\section{Unix}
В Unix планировщик предоставляет процессор каждому процессу системы на небольшой период времени, после чего производит переключение на следующий процесс. Этот период называется квантом времени.

\textbf{Переключение контекста} - на самом низком уровне планировщик заставляет процессор производить переключения от одного процесса к другому.

Классическое ядро UNIX является строго невытесняемым. Это означает, что если процесс выполняется в режиме ядра, то ядро не заставит этот процесс уступить процессорное время какому-либо более приоритетному процессу. Выполняющийся процесс может освободить процессор в случае своего блокирования в ожидании ресурса, иначе он может быть вытеснен при переходе в режим задачи. Такая реализация ядра позволяет решить множество проблем синхронизации, связанных с доступом нескольких процессов к одним и тем же структурам данных ядра.


\end{document}
