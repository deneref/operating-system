\documentclass[12pt]{report}
\usepackage[utf8]{inputenc}
\usepackage[russian]{babel}
%\usepackage[14pt]{extsizes}
\usepackage{listings}
\usepackage{xcolor}

% Для листинга кода:
\lstset{ %
	language= C,                 % выбор языка для подсветки 
	basicstyle=\small\sffamily, % размер и начертание шрифта для подсветки кода
	numbers=left,               % где поставить нумерацию строк (слева\справа)
	numberstyle=\tiny,           % размер шрифта для номеров строк
	stepnumber=1,                   % размер шага между двумя номерами строк
	numbersep=5pt,                % как далеко отстоят номера строк от подсвечиваемого кода
	showspaces=false,            % показывать или нет пробелы специальными отступами
	showstringspaces=false,      % показывать или нет пробелы в строках
	showtabs=false,             % показывать или нет табуляцию в строках            
	tabsize=2,                 % размер табуляции по умолчанию равен 2 пробелам
	captionpos=t,              % позиция заголовка вверху [t] или внизу [b] 
	breaklines=true,           % автоматически переносить строки (да\нет)
	breakatwhitespace=false, % переносить строки только если есть пробел
	escapeinside={\#*}{*)}   % если нужно добавить комментарии в коде
}

\definecolor{comment}{RGB}{0,128,0} % dark green
\definecolor{string}{RGB}{255,0,0}  % red
\definecolor{keyword}{RGB}{0,0,255} % blue

\lstdefinestyle{CStyle}{
	commentstyle=\color{comment},
	stringstyle=\color{string},
	keywordstyle=\color{keyword},
	basicstyle=\footnotesize\ttfamily,
	numbers=left,
	numberstyle=\tiny,
	numbersep=5pt,
	frame=lines,
	breaklines=true,
	prebreak=\raisebox{0ex}[0ex][0ex]{\ensuremath{\hookleftarrow}},
	showstringspaces=false,
	upquote=true,
	tabsize=2,
}

% Для измененных титулов глав:
\usepackage{titlesec, blindtext, color} % подключаем нужные пакеты
\definecolor{gray75}{gray}{0.75} % определяем цвет
\newcommand{\hsp}{\hspace{20pt}} % длина линии в 20pt
% titleformat определяет стиль
\titleformat{\chapter}[hang]{\Huge\bfseries}{\thechapter\hsp\textcolor{gray75}{|}\hsp}{0pt}{\Huge\bfseries}

%отступы по краям
\usepackage{geometry}
\geometry{verbose, a4paper,tmargin=2cm, bmargin=2cm, rmargin=1.5cm, lmargin = 3cm}
% межстрочный интервал
\usepackage{setspace}
\onehalfspacing
\usepackage{float}
% plot
\usepackage{pgfplots}
\usepackage{filecontents}
\usepackage{amsmath}
\usepackage{tikz,pgfplots}
\usetikzlibrary{datavisualization}
\usetikzlibrary{datavisualization.formats.functions}

\usepackage{graphicx}
\graphicspath{{src/}}
\DeclareGraphicsExtensions{.pdf,.png,.jpg}

\usepackage{geometry}
\geometry{verbose, a4paper,tmargin=2cm, bmargin=2cm, rmargin=1.5cm, lmargin = 3cm}
\usepackage{indentfirst}
\setlength{\parindent}{1.4cm}

\usepackage{titlesec}
\titlespacing{\chapter}{0pt}{12pt plus 4pt minus 2pt}{0pt}


\begin{document}
%\def\chaptername{} % убирает "Глава"
\begin{titlepage}
	\centering
	{\scshape\LARGE МГТУ им. Баумана \par}
	\vspace{3cm}
	{\scshape\Large Лабораторная работа №6\par}
	\vspace{0.5cm}	
	{\scshape\Large По курсу: "Операционные системы"\par}
	\vspace{1.5cm}
	\centering
	 {\huge\bfseries Реализация монитора Хоара «Читатели-писатели» под ОС Windows.\par}
	\vspace{2cm}
	\Large Работу выполнил: Мокеев Даниил, ИУ7-56\par
	\vspace{0.5cm}
	\Large Преподаватель:  Рязанова Н.Ю.\par

	\vfill
	\large \textit {Москва, 2020} \par
\end{titlepage}


\newpage

\section{Листинг кода алгоритмов}

\begin{lstlisting}[label=one,caption = Реализация монитора Хоара, style = CStyle]
#include <stdio.h>
#include <stdbool.h>
#include <windows.h>

#define READERS 5
#define WRITERS 3
#define WRITERS_ITTER 4
#define SLEEP_TIME 200
#define READER_BORDER "\t\t\t\t\t\t"

HANDLE writers[WRITERS];
HANDLE readers[READERS];

HANDLE mutex;
HANDLE can_read;
HANDLE can_write;
volatile LONG waiting_writers = 0;
volatile LONG waiting_readers = 0;
volatile LONG active_readers = 0;
bool is_writer_active = false;

int value = 0;

void start_read(void) {
	InterlockedIncrement(&waiting_readers);
	if (is_writer_active || WaitForSingleObject(can_write, 0) == WAIT_OBJECT_0) {
		WaitForSingleObject(can_read, INFINITE);
	}
	
	WaitForSingleObject(mutex, INFINITE);
	
	InterlockedDecrement(&waiting_readers);
	InterlockedIncrement(&active_readers);
	
	SetEvent(can_read);
	
	ReleaseMutex(mutex);
}

void stop_read(void) {
	InterlockedDecrement(&active_readers);
	
	if (waiting_readers == 0) {
		SetEvent(can_write);
	}
}

DWORD WINAPI reader(LPVOID lpParams) {
	while (value < 3 * WRITERS_ITTER) {
		start_read();
		printf(READER_BORDER"Reader #%ld; read value: %d\n", (int) lpParams, value);
		stop_read();
		Sleep(SLEEP_TIME);
	}
	
	return EXIT_SUCCESS;
}

void start_write(void) {
	InterlockedIncrement(&waiting_writers);
	if (is_writer_active || active_readers > 0) {
		WaitForSingleObject(can_write, INFINITE);
	}
	
	InterlockedDecrement(&waiting_writers);
	is_writer_active = true;
	ResetEvent(can_write);
}

void stop_write(void) {
	is_writer_active = false;
	
	if (!waiting_writers) {
		SetEvent(can_read);
	} else {
		SetEvent(can_write);
	}
}

DWORD WINAPI writer(LPVOID lpParams) {
	int i = 0;
	for (int i = 0; i < WRITERS_ITTER; ++i) {
		start_write();
		
		value++;
		printf("Writer #%ld wrote value: %ld\n", (int) lpParams, value);
		
		stop_write();
		Sleep(SLEEP_TIME);
	}
	
	return EXIT_SUCCESS;
}

int init_handles(void) {
	if ((mutex = CreateMutex(NULL, FALSE, NULL)) == NULL) {
		perror("error while CreateMutex");
		return EXIT_FAILURE;
	}
	
	if ((can_read = CreateEvent(NULL, TRUE, TRUE, NULL)) == NULL) {
		perror("error while CreateEvent can_read");
		return EXIT_FAILURE;
	}
	if ((can_write = CreateEvent(NULL, FALSE, TRUE, NULL)) == NULL) {
		perror("error while CreateEvent can_write");
		return EXIT_FAILURE;
	}
	
	return EXIT_SUCCESS;
}

int create_threads(HANDLE *threads, int threads_count, DWORD (*on_thread)(LPVOID)) {
	for (int i = 0; i < threads_count; ++i) {
		if ((threads[i] = CreateThread(NULL, 0, on_thread, (LPVOID) i, 0, NULL)) == NULL) {
			perror("error while CreateThread");
			return EXIT_FAILURE;
		}
	}
	
	return EXIT_SUCCESS;
}

int main(void) {
	setbuf(stdout, NULL);
	
	int rc = EXIT_SUCCESS;
	
	if ((rc = init_handles()) != EXIT_SUCCESS
	|| (rc = create_threads(writers, WRITERS, writer)) != EXIT_SUCCESS
	|| (rc = create_threads(readers, READERS, reader)) != EXIT_SUCCESS) {
		return rc;
	}
	
	WaitForMultipleObjects(WRITERS, writers, TRUE, INFINITE);
	WaitForMultipleObjects(READERS, readers, TRUE, INFINITE);
	
	CloseHandle(mutex);
	CloseHandle(can_read);
	CloseHandle(can_write);
	
	return rc;
}
\end{lstlisting}
\begin{figure}[H]
	\centering{\includegraphics[scale=1]{example.png}} 
	\caption{Пример работы программы.}
	\label{ris:1}
\end{figure}



\end{document}
