\documentclass[12pt]{report}
\usepackage[utf8]{inputenc}
\usepackage[russian]{babel}
%\usepackage[14pt]{extsizes}
\usepackage{listings}
\usepackage{nasm/lang}  % include custom language for NASM assembly.
\usepackage{nasm/style} % include custom style for NASM assembly.
\usepackage{xcolor}


% Для измененных титулов глав:
\usepackage{titlesec, blindtext, color} % подключаем нужные пакеты
\definecolor{gray75}{gray}{0.75} % определяем цвет
\newcommand{\hsp}{\hspace{20pt}} % длина линии в 20pt
% titleformat определяет стиль
\titleformat{\chapter}[hang]{\Huge\bfseries}{\thechapter\hsp\textcolor{gray75}{|}\hsp}{0pt}{\Huge\bfseries}

%отступы по краям
\usepackage{geometry}
\geometry{verbose, a4paper,tmargin=2cm, bmargin=2cm, rmargin=1.5cm, lmargin = 3cm}
% межстрочный интервал
\usepackage{setspace}
\onehalfspacing
\usepackage{float}
% plot
\usepackage{pgfplots}
\usepackage{filecontents}
\usepackage{amsmath}
\usepackage{tikz,pgfplots}
\usetikzlibrary{datavisualization}
\usetikzlibrary{datavisualization.formats.functions}

\usepackage{graphicx}
\graphicspath{{src/}}
\DeclareGraphicsExtensions{.pdf,.png,.jpg}

\usepackage{geometry}
\geometry{verbose, a4paper,tmargin=2cm, bmargin=2cm, rmargin=1.5cm, lmargin = 3cm}
\usepackage{indentfirst}
\setlength{\parindent}{1.4cm}

\usepackage{titlesec}
\titlespacing{\chapter}{0pt}{12pt plus 4pt minus 2pt}{0pt}


\begin{document}
	%\def\chaptername{} % убирает "Глава"
	\begin{titlepage}
		\centering
		{\scshape\LARGE МГТУ им. Баумана \par}
		\vspace{3cm}
		{\scshape\Large Лабораторная работа №2\par}
		\vspace{0.5cm}	
		{\scshape\Large По курсу: "Операционные системы"\par}
		\vspace{1.5cm}
		\centering
		{\huge\bfseries Защищенный режим.\par}
		\vspace{2cm}
		\Large Работу выполнил: Мокеев Даниил, ИУ7-56\par
		\vspace{0.5cm}
		\Large Преподаватель:  Рязанова Н.Ю.\par
		
		\vfill
		\large \textit {Москва, 2020} \par
	\end{titlepage}
	
	
	\newpage
	
	\chapter{Листинг кода алгоритмов}
	В данном разделе будут приведены листинги кода реализованных программ. 
	
	\begin{lstlisting}[label=one,caption = Процессы-сироты, language=nasm, style= nasm]
		;
		.386p
		
		descr struc
		limit  dw 0
		base_l dw 0
		base_m db 0
		attr_1 db 0
		arrt_2 db 0
		base_h db 0
		descr ends
		
		
		intr struc
		offs_l dw 0
		sel    dw 0
		rsrv   db 0
		attr   db 0
		offs_h dw 0
		intr ends
		
		
		pm_seg segment para public 'code' use32
		assume cs:pm_seg
		
		gdt             label byte
		gdt_null        descr <>
		gdt_data        descr <0FFFFh,0,0,92h,0CFh,0>
		gdt_code16      descr <rm_seg_size-1,0,0,98h,0,0>
		gdt_code32      descr <pm_seg_size-1,0,0,98h,0CFh,0>
		gdt_data32      descr <pm_seg_size-1,0,0,92h,0CFh,0>
		gdt_stack32     descr <stack_size-1,0,0,92h,0CFh,0>
		gdt_size=$-gdt
		
		gdtr            dw gdt_size-1
		dd ?
		
		; Селекторы сегментов
		sel_data        equ 8
		sel_code16      equ 16
		sel_code32      equ 24
		sel_data32      equ 32
		sel_stack32     equ 40
		
		idt             label byte
		trap1           intr 13 dup (<0,sel_code32,0,8Fh,0>)
		trap13          intr <0,sel_code32,0,8Fh,0>
		trap2           intr 18 dup (<0,sel_code32,0,8Fh,0>)
		int_time        intr <0,sel_code32,0,8Eh,0>
		int_keyboard    intr <0,sel_code32,0,8Eh,0>
		idt_size=$-idt
		
		idtr            dw idt_size-1
		dd ?
		
		rm_idtr         dw 3FFh,0,0
		
		hex             db 'h'
		hex_len=$-hex
		mb              db 'MB'
		mb_len=$-mb
		
		hello_msg       db 'DOS is in real mode now.$'
		pm_msg          db 'DOS switched to protected mode.'
		pm_msg_len=$-pm_msg
		tt_msg          db 'Timer ticks:      '
		tt_msg_len=$-tt_msg
		am_msg          db 'Available memory: '
		am_msg_len=$-am_msg
		esc_from_pr     db 'Press ESC to switch to real mode...'
		esc_from_pr_len=$-esc_from_pr
		ret_to_rm_msg   db 'DOS switched to real mode.$'
		
		scan2ascii      db 0,1Bh,'1','2','3','4','5','6','7','8','9','0','-','=',8
		db ' ','q','w','e','r','t','y','u','i','o','p','[',']','$'
		db ' ','a','s','d','f','g','h','j','k','l',';','""',0
		db '\','z','x','c','v','b','n','m',',','.','/',0,0,0,' ',0,0
		db 0,0,0,0,0,0,0,0,0,0,0,0
		
		attr1           db 3Fh
		attr2           db 4Fh
		screen_addr     dd 640
		timer           dd 0
		
		master          db 0
		slave           db 0
		
		; Макрос вывода строки в видеобуфер
		print_str macro msg,len,offset
		local print
		push   ebp
		mov    ecx,len
		mov    ebp,0B8000h
		add    ebp,offset
		xor    esi,esi
		mov    ah,attr2
		print:
		mov    al,byte ptr msg[esi]
		mov    es:[ebp],ax
		add    ebp,2
		inc    esi
		loop   print
		pop    ebp
		endm
		
		; Макрос отправки сигнал EOI контроллеру прерываний
		send_eoi macro
		mov    al,20h
		out    20h,al
		endm
		
		pm_start:
		; Установить сегменты защищенного режима
		mov    ax,sel_data
		mov    ds,ax
		mov    es,ax
		mov    ax,sel_stack32
		mov    ebx,stack_size
		mov    ss,ax
		mov    esp,ebx
		
		; Разрешить маскируемые прерывания
		sti
		
		; Вывести справочную информацию в видеобуфер
		print_str pm_msg,pm_msg_len,360
		print_str tt_msg,tt_msg_len,520
		print_str am_msg,am_msg_len,5*160+40
		print_str esc_from_pr,esc_from_pr_len,6*160+40
		
		call available_memory
		jmp    $
		
		; Обработчик исключения общей защиты
		exc13 proc
		pop    eax
		iret
		exc13 endp
		
		; Обработчик остальных исключений
		dummy_exc proc
		iret
		dummy_exc endp
		
		; Обработчик прерывания от системного таймера
		int_time_handler:
		push   eax
		push   ebp
		push   ecx
		push   dx
		
		; Загрузить счетчик
		mov    eax,timer
		
		; Вывести счетчик в видеобуфер
		mov    ebp,0B8000h
		mov    ecx,8
		add    ebp,530+2*(tt_msg_len)
		mov    dh,attr2
		main_loop:
		mov    dl,al
		and    dl,0Fh
		cmp    dl,10
		jl     less_than_10
		sub    dl,10
		add    dl,'A'
		jmp    print
		less_than_10:
		add    dl,'0'
		print:
		mov    es:[ebp],dx
		ror    eax,4
		sub    ebp,2
		loop   main_loop
		
		; Инкрементировать и сохранить счетчик
		inc    eax
		mov    timer,eax
		
		send_eoi
		pop    dx
		pop    ecx
		pop    ebp
		pop    eax
		
		iretd
		
		; Обработчик прерывания от клавиатуры
		int_keyboard_handler:
		push   eax
		push   ebx
		push   es
		push   ds
		
		; Получить из порта клавиатуры скан-код нажатой клавиши
		in     al,60h
		
		; Нажата клавиша ESC
		cmp    al,01h
		je     esc_pressed
		
		; Нажата необслуживаемая клавиша
		cmp    al,39h
		ja     skip_translate
		
		; Преобразовать скан-код в ASCII
		mov    bx,sel_data32
		mov    ds,bx
		mov    ebx,offset scan2ascii
		xlatb
		mov    bx,sel_data
		mov    es,bx
		mov    ebx,screen_addr
		
		; Нажата клавиша Backspace
		cmp    al,8
		je     bs_pressed
		
		; Вывести символ на экран
		mov    es:[ebx+0B8000h],al
		add    dword ptr screen_addr,2
		jmp    skip_translate
		
		bs_pressed:
		; Удалить символ
		mov    al,' '
		sub    ebx,2
		mov    es:[ebx+0B8000h],al
		mov    screen_addr,ebx
		
		skip_translate:
		; Разрешить работу клавиатуры
		in     al,61h
		or     al,80h
		out    61h,al
		
		send_eoi
		pop    ds
		pop    es
		pop    ebx
		pop    eax
		
		iretd
		
		esc_pressed:
		; Разрешить работу клавиатуры
		in     al,61h
		or     al,80h
		out    61h,al
		
		send_eoi
		pop    ds
		pop    es
		pop    ebx
		pop    eax
		
		; Запретить маскируемые прерывания
		cli
		
		; Вернуться в реальный режим
		db    0EAh
		dd    offset rm_return
		dw    sel_code16
		
		; Процедура определения доступного объема оперативной памяти
		available_memory proc
		push   ds
		
		mov    ax,sel_data
		mov    ds,ax
		
		; Пропустить первый мегабайт памяти
		mov    ebx,100001h
		; Установить проверочный байт
		mov    dl,0FFh
		; Установить максимальный объем оставшейся оперативной памяти
		mov    ecx,0FFEFFFFFh
		
		check:
		; Проверка сигнатуры
		mov    dh,ds:[ebx]
		mov    ds:[ebx],dl
		cmp    ds:[ebx],dl
		jnz    end_of_memory
		mov    ds:[ebx],dh
		inc    ebx
		loop   check
		
		end_of_memory:
		pop    ds
		xor    edx,edx
		mov    eax,ebx
		
		; Разделить на мегабайт
		mov    ebx,100000h
		div    ebx
		
		push   ecx
		push   dx
		push   ebp
		
		; Вывести объем памяти на экран
		mov    ebp,0B8000h
		mov    ecx,8
		add    ebp,5*160+2*(am_msg_len+7)+40
		mov    dh,attr2
		cycle:
		mov    dl,al
		and    dl,0Fh
		cmp    dl,10
		jl     number
		sub    dl,10
		add    dl,'A'
		jmp    print_m
		number:
		add    dl,'0'
		print_m:
		mov    es:[ebp],dx
		ror    eax,4
		
		sub    ebp,2
		loop   cycle
		sub    ebp,0B8000h
		
		pop    ebp
		pop    dx
		pop    ecx
		ret
		available_memory endp
		
		pm_seg_size=$-gdt
		pm_seg ends
		
		
		rm_seg segment para public 'code' use16
		assume cs:rm_seg,ds:pm_seg,ss:s_seg
		
		; Макрос очистки экрана
		cls macro
		mov    ax,3
		int    10h
		endm
		
		; Макрос печати строки
		print_str macro msg
		mov    ah,9
		mov    edx,offset msg
		int    21h
		endm
		
		rm_start:
		mov    ax,pm_seg
		mov    ds,ax
		
		cls
		
		mov    AX, 0B800h
		mov    ES, AX
		mov    DI, 200
		mov    cx, 24
		mov    ebx, offset hello_msg
		mov    ah, attr1
		mov    al, byte ptr [ebx]
		screen0:
		stosw
		inc    bx
		mov    al, byte ptr [ebx]
		loop   screen0
		
		
		; Вычислить базы для всех используемых дескрипторов сегментов
		xor    eax,eax
		mov    ax,rm_seg
		shl    eax,4
		mov    word ptr gdt_code16+2,ax
		shr    eax,16
		mov    byte ptr gdt_code16+4,al
		mov    ax,pm_seg
		shl    eax,4
		push   eax
		push   eax
		mov    word ptr gdt_code32+2,ax
		mov    word ptr gdt_stack32+2,ax
		mov    word ptr gdt_data32+2,ax
		shr    eax,16
		mov    byte ptr gdt_code32+4,al
		mov    byte ptr gdt_stack32+4,al
		mov    byte ptr gdt_data32+4,al
		
		; Вычислить линейный адрес GDT
		pop    eax
		add    eax,offset GDT
		mov    dword ptr gdtr+2,eax
		mov    word ptr gdtr,gdt_size-1
		
		; Установить регистр GDTR
		lgdt   fword ptr gdtr
		
		; Вычислить линейный адрес IDT
		pop    eax
		add    eax,offset idt
		mov    dword ptr idtr+2,eax
		mov    word ptr idtr,idt_size-1
		
		; Заполнить смещения в дескрипторах прерываний
		mov    eax,offset dummy_exc
		mov    trap1.offs_l,ax
		shr    eax,16
		mov    trap1.offs_h,ax
		mov    eax,offset exc13
		mov    trap13.offs_l,ax
		shr    eax,16
		mov    trap13.offs_h,ax
		mov    eax,offset dummy_exc
		mov    trap2.offs_l,ax
		shr    eax,16
		mov    trap2.offs_h,ax
		mov    eax,offset int_time_handler
		mov    int_time.offs_l,ax
		shr    eax,16
		mov    int_time.offs_h,ax
		mov    eax,offset int_keyboard_handler
		mov    int_keyboard.offs_l,ax
		shr    eax,16
		mov    int_keyboard.offs_h,ax
		
		; Cохранить маски ведущего и ведомого контроллеров прерываний
		in     al,21h
		mov    master,al
		in     al,0A1h
		mov    slave,al
		
		; Перепрограммировать ведущий контроллер прерываний
		mov    dx,20h
		mov    al,11h
		out    dx,al
		inc    dx
		mov    al,20h
		out    dx,al
		mov    al,4
		out    dx,al
		mov    al,1
		out    dx, al
		
		; Запретить все прерывания в ведущем контроллере, кроме IRQ0 и IRQ1
		mov    al,11111100b
		out    dx,al
		
		; Запретить все прерывания в ведомом контроллере
		mov    dx,0A1h
		mov    al,0FFh
		out    dx,al
		
		; Загрузить регистр IDTR
		lidt   fword ptr idtr
		
		; Открыть линию А20
		mov    al,0D1h
		out    64h,al
		mov    al,0DFh
		out    60h,al
		
		; Отключить маскируемые и немаскируемые прерывания
		cli
		in     al,70h
		or     al,80h
		out    70h,al
		
		; Перейти в защищенный режим установкой соответствующего бита регистра CR0
		mov    eax,cr0
		or     al,1
		mov    cr0,eax
		
		; Перейти в сегмент кода защищенного режима
		db     66h
		db     0EAh
		dd     offset pm_start
		dw     sel_code32
		
		rm_return:
		; Перейти в реальный режим сбросом соответствующего бита регистра CR0
		mov    eax,cr0
		and    al,0FEh
		mov    cr0,eax
		
		; Сбросить очередь и загрузить CS
		db     0EAh
		dw     $+4
		dw     rm_seg
		
		; Восстановить регистры для работы в реальном режиме
		mov    ax,pm_seg
		mov    ds,ax
		mov    es,ax
		mov    ax,s_seg
		mov    ss,ax
		mov    ax,stack_size
		mov    sp,ax
		
		; Инициализировать контроллер прерываний
		mov    al,11h
		out    20h,al
		mov    al,8
		out    21h,al
		mov    al,4
		out    21h,al
		mov    al,1
		out    21h,al
		
		; Восстановить маски контроллеров прерываний
		mov    al,master
		out    21h,al
		mov    al,slave
		out    0A1h,al
		
		; Загрузить таблицу дескрипторов прерываний реального режима
		lidt   fword ptr rm_idtr
		
		; Закрыть линию А20
		mov    al,0D1h
		out    64h,al
		mov    al,0DDh
		out    60h,al
		
		; Разрешить немаскируемые и маскируемые прерывания
		in     al,70h
		and    al,07FH
		out    70h,al
		sti
		
		;cls
		mov    AX, 0B800h
		mov    ES, AX
		mov    DI, 7*160+40
		mov    cx, 26
		mov    ebx, offset ret_to_rm_msg
		mov    ah, attr1
		mov    al, byte ptr [ebx]
		screen01:
		stosw
		inc    bx
		mov    al, byte ptr [ebx]
		loop   screen01
		
		mov    ah,4Ch
		int    21h
		rm_seg_size = $-rm_start
		rm_seg ends
		
		s_seg segment para stack 'stack'
		stack_start db 100h dup(?)
		stack_size=$-stack_start
		s_seg ends
		end    rm_start
	\end{lstlisting}
	
\end{document}
