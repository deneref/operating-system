\documentclass[12pt]{report}
\usepackage[utf8]{inputenc}
\usepackage[russian]{babel}
%\usepackage[14pt]{extsizes}
\usepackage{listings}
\usepackage{xcolor}

% Для листинга кода:
\lstset{ %
	language= C,                 % выбор языка для подсветки 
	basicstyle=\small\sffamily, % размер и начертание шрифта для подсветки кода
	numbers=left,               % где поставить нумерацию строк (слева\справа)
	numberstyle=\tiny,           % размер шрифта для номеров строк
	stepnumber=1,                   % размер шага между двумя номерами строк
	numbersep=5pt,                % как далеко отстоят номера строк от подсвечиваемого кода
	showspaces=false,            % показывать или нет пробелы специальными отступами
	showstringspaces=false,      % показывать или нет пробелы в строках
	showtabs=false,             % показывать или нет табуляцию в строках            
	tabsize=2,                 % размер табуляции по умолчанию равен 2 пробелам
	captionpos=t,              % позиция заголовка вверху [t] или внизу [b] 
	breaklines=true,           % автоматически переносить строки (да\нет)
	breakatwhitespace=false, % переносить строки только если есть пробел
	escapeinside={\#*}{*)}   % если нужно добавить комментарии в коде
}

\definecolor{comment}{RGB}{0,128,0} % dark green
\definecolor{string}{RGB}{255,0,0}  % red
\definecolor{keyword}{RGB}{0,0,255} % blue

\lstdefinestyle{CStyle}{
	commentstyle=\color{comment},
	stringstyle=\color{string},
	keywordstyle=\color{keyword},
	basicstyle=\footnotesize\ttfamily,
	numbers=left,
	numberstyle=\tiny,
	numbersep=5pt,
	frame=lines,
	breaklines=true,
	prebreak=\raisebox{0ex}[0ex][0ex]{\ensuremath{\hookleftarrow}},
	showstringspaces=false,
	upquote=true,
	tabsize=2,
}

% Для измененных титулов глав:
\usepackage{titlesec, blindtext, color} % подключаем нужные пакеты
\definecolor{gray75}{gray}{0.75} % определяем цвет
\newcommand{\hsp}{\hspace{20pt}} % длина линии в 20pt
% titleformat определяет стиль
\titleformat{\chapter}[hang]{\Huge\bfseries}{\thechapter\hsp\textcolor{gray75}{|}\hsp}{0pt}{\Huge\bfseries}

%отступы по краям
\usepackage{geometry}
\geometry{verbose, a4paper,tmargin=2cm, bmargin=2cm, rmargin=1.5cm, lmargin = 3cm}
% межстрочный интервал
\usepackage{setspace}
\onehalfspacing
\usepackage{float}
% plot
\usepackage{pgfplots}
\usepackage{filecontents}
\usepackage{amsmath}
\usepackage{tikz,pgfplots}
\usetikzlibrary{datavisualization}
\usetikzlibrary{datavisualization.formats.functions}

\usepackage{graphicx}
\graphicspath{{src/}}
\DeclareGraphicsExtensions{.pdf,.png,.jpg}

\usepackage{geometry}
\geometry{verbose, a4paper,tmargin=2cm, bmargin=2cm, rmargin=1.5cm, lmargin = 3cm}
\usepackage{indentfirst}
\setlength{\parindent}{1.4cm}

\usepackage{titlesec}
\titlespacing{\chapter}{0pt}{12pt plus 4pt minus 2pt}{0pt}


\begin{document}
	%\def\chaptername{} % убирает "Глава"
	\begin{titlepage}
		\centering
		{\scshape\LARGE МГТУ им. Баумана \par}
		\vspace{3cm}
		{\scshape\Large Лабораторная работа №4\par}
		\vspace{0.5cm}	
		{\scshape\Large По курсу: "Операционные системы"\par}
		\vspace{1.5cm}
		\centering
		{\huge\bfseries Процессы. \par}
		\centering
		{\huge\bfseries Системные вызовы fork() и exec().\par}
		\vspace{2cm}
		\Large Работу выполнил: Мокеев Даниил, ИУ7-56\par
		\vspace{0.5cm}
		\Large Преподаватель:  Рязанова Н.Ю.\par
		
		\vfill
		\large \textit {Москва, 2020} \par
	\end{titlepage}
	
	
	\newpage
	
	\section{Листинг кода алгоритмов}
	В данном разделе будут приведены листинги кода реализованных программ. 
	
	\begin{lstlisting}[label=one,caption = Процессы-сироты, style = CStyle]
		#include <stdio.h>
		#include <stdlib.h>
		
		int main(){
			int child_1, child_2;
			child_1 = fork();
			if(child_1 == -1){
				perror("Coulnd't fork child #1");
				exit(1);
			}
			if (child_1 == 0){
				sleep(1);
				printf("Child #1: pid=%d; group=%d; ppid=%d\n",
				getpid(), getpgrp(), getppid());
				
				return 0;
			}
			if (child_1 > 0){
				child_2 = fork();
				if(child_2 == -1){
					perror("Coulnd't fork child #2");
					exit(1);
				}
				if (child_2 == 0){
					printf("\nChild #2: pid=%d; group=%d; ppid=%d\n",
					getpid(), getpgrp(), getppid());
					
					return 0;
				}else{
					printf("Parent: pid=%d; group=%d; ppid=%d\n",
					getpid(), getpgrp(), getppid());
					
					return 0;
				}
			}
		}
	\end{lstlisting}
	\begin{figure}[H]
		\centering{\includegraphics[scale=1]{task1.png}} 
		\caption{Пример работы программы №1}
		\label{ris:task1}
	\end{figure}
	\newpage
	\begin{lstlisting}[label=two,caption = Использование вызова wait(), style = CStyle]
		#include <stdio.h>
		#include <stdlib.h>
		#include <sys/types.h>
		#include <sys/wait.h>
		
		int main(){
			int child_1, child_2;
			child_1 = fork();
			if(child_1 == -1){
				perror("Coulnd't fork child #1");
				exit(1);
			}
			if (child_1 == 0){
				sleep(1);
				printf("Child #1: pid=%d; group=%d; ppid=%d\n",
				getpid(), getpgrp(), getppid());
				
				return 0;
			}
			if (child_1 > 0){
				child_2 = fork();
				if(child_2 == -1){
					perror("Coulnd't fork child #2");
					exit(1);
				}
				if (child_2 == 0){
					printf("\nChild #2: pid=%d; group=%d; ppid=%d\n",
					getpid(), getpgrp(), getppid());
					
					return 0;
				}else{	
					printf("Parent: pid=%d; group=%d; ppid=%d\n",
					getpid(), getpgrp(), getppid());
					
					pid_t child_pid;
					int status;
					
					child_pid = wait(&status);
					if (WIFEXITED(status))
					printf("Parent: child %d finished with code %d\n",
					child_pid, WEXITSTATUS(status) );
					else if (WIFSTOPPED(status))
					printf("Parent: child %d finished with code %d\n",
					child_pid, WSTOPSIG(status) );
					return 0;
				}
			}
			
		}
	\end{lstlisting}
	\begin{figure}[H]
		\centering{\includegraphics[scale=0.9]{task2.png}} 
		\caption{Пример работы программы №2}
		\label{ris:task2}
	\end{figure}
	\newpage
	\begin{lstlisting}[label=three,caption =  Использование вызова exec(), style = CStyle]
		#include <stdio.h>
		#include <stdlib.h>
		#include <unistd.h>
		
		int main(){
			int child_1, child_2;
			child_1 = fork();
			if(child_1 == -1){
				perror("Coulnd't fork child #1");
				exit(1);
			}
			if (child_1 == 0){
				sleep(1);
				printf("\nChild #1 executes ps -al\n\n");
				if(execlp("ps", "ps", "-al", (char*)NULL) == -1){
					printf("Couldn't exec ps command\n");
					exit(1);
				}
				
				return 0;
			}
			if (child_1 > 0){
				child_2 = fork();
				if(child_2 == -1){
					perror("Coulnd't fork child #2");
					exit(1);
				}
				if (child_2 == 0){
					printf("\nChild #2 executes ls -l\n\n");
					if(execlp("ls", "ls", "-l", (char*)NULL) == -1){
						printf("Couldn't exec ps command\n");
						exit(1);
					}
					return 0;
				}else{
					printf("\nParent: pid=%d; group=%d; ppid=%d\n",
					getpid(), getpgrp(), getppid());
					
					pid_t child_pid;
					int status;
					
					//waiting for the second child to finish
					child_pid = wait(&status);
					if (WIFEXITED(status))
					printf("\nParent: child with pid = %d finished with code %d\n",
					child_pid, WEXITSTATUS(status) );
					else if (WIFSTOPPED(status))
					printf("\nParent: child %d finished with code %d\n",
					child_pid, WSTOPSIG(status) );
					
					//waiting for the first child to finish	 
					child_pid = wait(&status);
					if (WIFEXITED(status))
					printf("\nParent: child with pid = %d finished with code %d\n",
					child_pid, WEXITSTATUS(status) );
					else if (WIFSTOPPED(status))
					printf("\nParent: child %d finished with code %d\n",
					child_pid, WSTOPSIG(status) );
					
					return 0;
				}
			}
		}
	\end{lstlisting}
	\begin{figure}[H]
		\centering{\includegraphics[scale=0.5]{task3.png}} 
		\caption{Пример работы программы №3}
		\label{ris:task3}
	\end{figure}
	\newpage
	\begin{lstlisting}[label=four,caption = Передача сообщений с помощью программного канала, style = CStyle]
		/*
		exchanging messages with parent
		*/
		#include <stdio.h>
		#include <stdlib.h>
		#include <unistd.h>
		#define MESSAGE_SIZE 32
		
		int main(){
			int child_1, child_2;
			
			//initializing pipe
			int fd[2];
			if (pipe(fd) == -1){
				printf("Coundn't create a pipe\n");
				exit(1);
			}
			
			child_1 = fork();
			if(child_1 == -1){
				perror("Coulnd't fork child #1");
				exit(1);
			}
			if (child_1 == 0){
				
				
				close(fd[0]);
				if (write(fd[1], "Hello from child #1, parent!\n",
				MESSAGE_SIZE) > 0)
				printf("Child #1 sent a greeting to the parent\n");
				return 0;
			}
			if (child_1 > 0){
				child_2 = fork();
				if(child_2 == -1){
					perror("Coulnd't fork child #2");
					exit(1);
				}
				if (child_2 == 0){
					sleep(1);
					close(fd[0]);
					if (write(fd[1], "Hello from child #2, parent!\n",
					MESSAGE_SIZE) > 0)
					printf("Child #2 sent a greeting to the parent\n");
					return 0;
				}else{
					
					char msg1[MESSAGE_SIZE], msg2[MESSAGE_SIZE];
					close(fd[1]);
					read(fd[0], msg1, MESSAGE_SIZE);
					read(fd[0], msg2, MESSAGE_SIZE);
					
					printf("Parent read a message from his children: \n%s\n%s",
					msg1, msg2); 
					
					
					
					pid_t child_pid;
					int status;
					
					//waiting for the second child to finish
					child_pid = wait(&status);
					if (WIFEXITED(status))
					printf("\nParent: child with pid = %d finished with code %d\n",
					child_pid, WEXITSTATUS(status) );
					else if (WIFSTOPPED(status))
					printf("\nParent: child %d finished with code %d\n",
					child_pid, WSTOPSIG(status) );
					
					//waiting for the first child to finish	 
					child_pid = wait(&status);
					if (WIFEXITED(status))
					printf("\nParent: child with pid = %d finished with code %d\n",
					child_pid, WEXITSTATUS(status) );
					else if (WIFSTOPPED(status))
					printf("\nParent: child %d finished with code %d\n",
					child_pid, WSTOPSIG(status) );
					
					return 0;
				}
			}
		}
		
		
		
	\end{lstlisting}
	\begin{figure}[H]
		\centering{\includegraphics[scale=0.7]{task4.png}} 
		\caption{Пример работы программы №4}
		\label{ris:task4}
	\end{figure}
	\newpage
	\begin{lstlisting}[label=five,caption = Установка своего обработчика сигнала, style = CStyle]
		/*
		DIY sig handler
		*/
		#include <stdio.h>
		#include <stdlib.h>
		#include <unistd.h>
		#include <signal.h>
		#include <stdbool.h>
		#include <string.h>
		
		#define MESSAGE_SIZE 64
		#define SLEEP_TIME 3
		
		//flag is set to true if signal has been cought
		bool flag = false;
		
		void mySignalHandler(int snum){
			printf("\nHandlig signal snum = %d in prosses...\n", snum);
			printf("Done!\n");
			flag = true;
		}
		
		int main(){
			int child_1, child_2;
			signal(SIGINT, mySignalHandler);
			
			//initializing pipe
			int fd[2];
			if (pipe(fd) == -1){
				printf("Coundn't create a pipe\n");
				exit(1);
			}
			
			child_1 = fork();
			if(child_1 == -1){
				perror("Coulnd't fork child #1");
				exit(1);
			}
			if (child_1 == 0){
				close(fd[1]);
				sleep(SLEEP_TIME);
				if (flag){
					char msg[MESSAGE_SIZE];
					if (read(fd[0], msg, MESSAGE_SIZE) > 0){
						printf("Child #1 read from parent %s\n", msg);
					}
				}
				
				return 0;
			}
			if (child_1 > 0){
				child_2 = fork();
				if(child_2 == -1){
					perror("Coulnd't fork child #2");
					exit(1);
				}
				if (child_2 == 0){
					close(fd[1]);
					
					sleep(SLEEP_TIME);
					if (flag){
						char msg[MESSAGE_SIZE];
						if (read(fd[0], msg, MESSAGE_SIZE) > 0){
							printf("Child #2 read from parent %s\n", msg);
						}
					}
					
					return 0;
				}else{
					close(fd[0]);
					printf("Parent's waiting for Ctrl+C being pressed to send messages from children\n");
					sleep(SLEEP_TIME);
					
					if (flag){
						//writing in pipe if we cought the signal
						if (write(fd[1], "Hello, my child!\n", MESSAGE_SIZE) > 0)
						printf("Parent sent his first greeting\n");
						if (write(fd[1], "Hello again, my child!\n", MESSAGE_SIZE) > 0)
						printf("Parent sent his second greeting\n");
					}
					
					pid_t child_pid;
					int status;
					
					//waiting for the second child to finish
					child_pid = wait(&status);
					if (WIFEXITED(status))
					printf("Parent: child with pid = %d finished with code %d\n",
					child_pid, WEXITSTATUS(status) );
					else if (WIFSTOPPED(status))
					printf("Parent: child %d finished with code %d\n",
					child_pid, WSTOPSIG(status) );
					
					//waiting for the first child to finish	 
					child_pid = wait(&status);
					if (WIFEXITED(status))
					printf("Parent: child with pid = %d finished with code %d\n",
					child_pid, WEXITSTATUS(status) );
					else if (WIFSTOPPED(status))
					printf("Parent: child %d finished with code %d\n",
					child_pid, WSTOPSIG(status) );
					return 0;
				}
			}
		}
	\end{lstlisting}
	\begin{figure}[H]
		\centering{\includegraphics[scale=1]{task5-1.png}} 
		\caption{Пример работы программы №5: был подан сигнал}
		\label{ris:task5-1}
	\end{figure}
	\begin{figure}[H]
		\centering{\includegraphics[scale=1]{task5-2.png}} 
		\caption{Пример работы программы №5: сигнал не был подан}
		\label{ris:task5-2}
	\end{figure}
	\newpage
	
	
	
\end{document}